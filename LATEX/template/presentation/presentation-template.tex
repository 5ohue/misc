\documentclass[t]{beamer}  % [t], [c], или [b] --- вертикальное выравнивание на слайдах (верх, центр, низ)
%\documentclass[handout]{beamer} % Раздаточный материал (на слайдах всё сразу)
%\documentclass[aspectratio=169]{beamer} % Соотношение сторон

% С сайта https://hartwork.org/beamer-theme-matrix/
\usetheme{Antibes}% Тема оформления
\usecolortheme{seagull} % Цветовая схема

%%% Работа с русским языком
\usepackage{cmap}					% поиск в PDF
\usepackage{mathtext} 				% русские буквы в формулах
\usepackage[T2A]{fontenc}			% кодировка
\usepackage[utf8]{inputenc}			% кодировка исходного текста
\usepackage[english,russian]{babel}	% локализация и переносы

%% Beamer по-русски
\newtheorem{rtheorem}{Теорема}
\newtheorem{rproof}{Доказательство}
\newtheorem{rexample}{Пример}

%%% Дополнительная работа с математикой
\usepackage{amsmath,amsfonts,amssymb,amsthm,mathtools} % AMS
\usepackage{icomma} % "Умная" запятая: $0,2$ --- число, $0, 2$ --- перечисление

%% Номера формул
%\mathtoolsset{showonlyrefs=true} % Показывать номера только у тех формул, на которые есть \eqref{} в тексте.
%\usepackage{leqno} % Нумерация формул слева

%% Свои команды
\DeclareMathOperator{\sgn}{\mathop{sgn}}

%% Перенос знаков в формулах (по Львовскому)
\newcommand*{\hm}[1]{#1\nobreak\discretionary{}
	{\hbox{$\mathsurround=0pt #1$}}{}}

%%% Работа с картинками
\usepackage{graphicx}  % Для вставки рисунков
\usepackage{wrapfig} % Обтекание рисунков текстом

%%% Работа с таблицами
\usepackage{array,tabularx,tabulary,booktabs} % Дополнительная работа с таблицами
\usepackage{longtable}  % Длинные таблицы
\usepackage{multirow} % Слияние строк в таблице

%%% Программирование
\usepackage{etoolbox} % логические операторы

%%% Другие пакеты
\usepackage{lastpage} % Узнать, сколько всего страниц в документе.
\usepackage{soul} % Модификаторы начертания
\usepackage{csquotes} % Еще инструменты для ссылок
%\usepackage[style=authoryear,maxcitenames=2,backend=biber,sorting=nty]{biblatex}
\usepackage{multicol} % Несколько колонок

\title{Идеальный Ужин}
\author{5ohue}
\date{\today}

\begin{document}
	
	\frame[plain]{\titlepage}
	
	\section{Закуски}
	
	\begin{frame}\label{lab}
		\frametitle{\insertsection}
		
		\begin{itemize}
			\item \href{https://www.russianfood.com/recipes/recipe.php?rid=144672}{\alert{Бутерброды со шпротами}};
			\item \href{https://www.russianfood.com/recipes/recipe.php?rid=142143}{\alert{Салат с курицей и корейской морковкой}};
			\item Куриные рулетики с сыром;
			\item Cалат Цезарь;
			\item Омлетный рулет с сыром.
		\end{itemize}
		
	\end{frame}
	
	\section{Основное блюдо}
	
	\begin{frame}
		\frametitle{\insertsection}
		
		\begin{block}{Мясо}
			Гостям будет подано мясо:
			\pause
			\begin{itemize}
				\item Шашлыки; \pause
				\item Рыба; \pause
				\item Курица.
			\end{itemize}
		\end{block}
		
		\begin{block}{Гарниры}
			А также гарниры:
			\pause
			\begin{itemize}
				\item Печеный картофель; \pause
				\item Рис.
			\end{itemize}
		\end{block}
		
		
	\end{frame}
	
	\begin{frame}
		\begin{block}{Убедитесь, что вы ничего не пропустили!}
			\hyperlink{lab}{\beamerbutton{переход к 1 слайду}} 
		\end{block}
	\end{frame}
	
\end{document}
